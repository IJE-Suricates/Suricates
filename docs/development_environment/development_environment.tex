%%%%%%%%%%%%%%%%%%%%%%%%%%%%%%%%%%%%%%%%%
% Thin Sectioned Essay
% LaTeX Template
% Version 1.0 (3/8/13)
%
% This template has been downloaded from:
% http://www.LaTeXTemplates.com
%
% Original Author:
% Nicolas Diaz (nsdiaz@uc.cl) with extensive modifications by:
% Vel (vel@latextemplates.com)
%
% License:
% CC BY-NC-SA 3.0 (http://creativecommons.org/licenses/by-nc-sa/3.0/)
%
%%%%%%%%%%%%%%%%%%%%%%%%%%%%%%%%%%%%%%%%%

%----------------------------------------------------------------------------------------
%	PACKAGES AND OTHER DOCUMENT CONFIGURATIONS
%----------------------------------------------------------------------------------------

\documentclass[a4paper, 11pt]{article} % Font size (can be 10pt, 11pt or 12pt) and paper size (remove a4paper for US letter paper)

\usepackage[utf8]{inputenc} % Set utf8 code
\usepackage[protrusion=true,expansion=true]{microtype} % Better typography
\usepackage{graphicx} % Required for including pictures
\usepackage{wrapfig} % Allows in-line images

\usepackage{mathpazo} % Use the Palatino font
\usepackage[T1]{fontenc} % Required for accented characters
\linespread{1.05} % Change line spacing here, Palatino benefits from a slight increase by default

\makeatletter
\renewcommand\@biblabel[1]{\textbf{#1.}} % Change the square brackets for each bibliography item from '[1]' to '1.'
\renewcommand{\@listI}{\itemsep=0pt} % Reduce the space between items in the itemize and enumerate environments and the bibliography

\renewcommand{\maketitle}{ % Customize the title - do not edit title and author name here, see the TITLE block below
\begin{center} % Right align
{\LARGE\@title} % Increase the font size of the title

\vspace{20pt} % Some vertical space between the title and author name

\end{center}
}

%----------------------------------------------------------------------------------------
%	TITLE
%----------------------------------------------------------------------------------------

\title{\textbf{Ambiente de desenvolvimento}} % Title

%----------------------------------------------------------------------------------------

\begin{document}

\maketitle % Print the title section

%----------------------------------------------------------------------------------------
%	DOC BODY
%----------------------------------------------------------------------------------------

\section*{Escolha das ferramentas}

A seguir são apresentados as ferramentas que serão utilizadas no ambiente de desenvolvimento do jogo.

\begin{enumerate}
\item \textbf{Linguagem de programação}

A linguagem escolhida foi o C++, pois esta linguagem predomina da indústria de jogos. Como alguns jogos são altamente dinâmicos, é necessário que a arquitetura do jogo esteja voltada para o desempenho, e neste quesito o C++ é praticamente imbatível, tornando assim o C++ uma linguagem de programção tão difundida no cenário de jogos.

\item \textbf{Compilador}

O compilador escolhido foi o g++ por este ser o compilador \textit{GNU} de sistemas Unix e Linux para linguagem C++.
\item \textbf{Linguagem de script}

A definir.
\item \textbf{Editor de texto}

Foi definido o uso do Sublime Text 3 pela equipe de desenvolvimento, este foi escolhido pelo domínio que os desenvolvedores do time possuem com tal editor e por este ser simples, leve e eficiente no desenvolvimento.
\item \textbf{Depurador}

O depurador selecionado foi o  gdb, que é o depurador do linux (\textit{GNU Debugger}) em modo de texto.
\item \textbf{Ferramenta de controle de versão}

A ferramenta de controle de versão que será utilizada é o git, pois esta é livre e é uma das mais difundidas no mercado.
\item \textbf{API gráfica}

SDL2 (\textit{Simple Directmedia Layer}): é uma biblioteca multiplataforma para desenvolvimento de jogos.
\item \textbf{API de áudio}

A definir.
\item \textbf{APIs para manipulação de arquivos}

SDL\_image.
\item \textbf{Sistema operacional}

O sistema operaional definido é o Ubuntu. Este sistema opereracional foi escolhido por ser livre e por todos os desenvolvedores já serem familharizados com este.

\end{enumerate}

\end{document}
