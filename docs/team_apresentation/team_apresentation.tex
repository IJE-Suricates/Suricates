%%%%%%%%%%%%%%%%%%%%%%%%%%%%%%%%%%%%%%%%%
% Thin Sectioned Essay
% LaTeX Template
% Version 1.0 (3/8/13)
%
% This template has been downloaded from:
% http://www.LaTeXTemplates.com
%
% Original Author:
% Nicolas Diaz (nsdiaz@uc.cl) with extensive modifications by:
% Vel (vel@latextemplates.com)
%
% License:
% CC BY-NC-SA 3.0 (http://creativecommons.org/licenses/by-nc-sa/3.0/)
%
%%%%%%%%%%%%%%%%%%%%%%%%%%%%%%%%%%%%%%%%%

%----------------------------------------------------------------------------------------
%	PACKAGES AND OTHER DOCUMENT CONFIGURATIONS
%----------------------------------------------------------------------------------------

\documentclass[a4paper, 11pt]{article} % Font size (can be 10pt, 11pt or 12pt) and paper size (remove a4paper for US letter paper)

\usepackage[utf8]{inputenc} % Set utf8 code
\usepackage[protrusion=true,expansion=true]{microtype} % Better typography
\usepackage{graphicx} % Required for including pictures
\usepackage{wrapfig} % Allows in-line images

\usepackage{mathpazo} % Use the Palatino font
\usepackage[T1]{fontenc} % Required for accented characters
\linespread{1.05} % Change line spacing here, Palatino benefits from a slight increase by default

\makeatletter
\renewcommand\@biblabel[1]{\textbf{#1.}} % Change the square brackets for each bibliography item from '[1]' to '1.'
\renewcommand{\@listI}{\itemsep=0pt} % Reduce the space between items in the itemize and enumerate environments and the bibliography

\renewcommand{\maketitle}{ % Customize the title - do not edit title and author name here, see the TITLE block below
\begin{center} % Right align
{\LARGE\@title} % Increase the font size of the title

\vspace{20pt} % Some vertical space between the title and author name

\end{center}
}

%----------------------------------------------------------------------------------------
%	TITLE
%----------------------------------------------------------------------------------------

\title{\textbf{Apresentação da Equipe}} % Title

%----------------------------------------------------------------------------------------

\begin{document}

\maketitle % Print the title section

%----------------------------------------------------------------------------------------
%	DOC BODY
%----------------------------------------------------------------------------------------

\section*{Informações da Equipe}

Este documento irá apresentar a equipe que estará desenvolvendo o jogo \textbf{Voidcrawlers} durante esse semestre.

\begin{itemize}
\item \textbf{Localização:} A equipe está localizada na Universidade de Brasília nos Campus Darcy Ribeiro e Gama, sendo que no Darcy estarão os membros responsáveis por criar a arte e música do jogo e um desenvolvedor, no Gama estarão presentes mais três desenvolvedores.

\item \textbf{Nome fantasia:} Tiamat.
\item \textbf{Histórico:} Os membros da equipe não possuem juntos outros trabalhos anteriores.
\end{itemize}

\section*{Membros da Equipe}

A seguir são apresentados cada um dos membros da equipe:

\begin{enumerate}
\item Álex Silva Mesquita

\begin{itemize}
\item \textbf{Função:} Game Design da equipe
\item \textbf{Histórico:} 21 Anos, Cursando 8º Semestre de Engenharia de Software no campus Gama, ainda sem nenhuma experiência com o desenvolvimento de jogos. Conhecimento de linguagens C/C++, Java, Groovy, e um pouco de Ruby e Python.
\end{itemize}

\item Jefferson Nunes de Sousa Xavier

\begin{itemize}
\item \textbf{Função:} Gerente da equipe de desenvolvedores
\item \textbf{Histórico:} 21 Anos, Cursando 8º Semestre de Engenharia de Software no campus Gama. Ainda sem nenhuma experiência com o desenvolvimento de jogos. Conhecimento de linguagens C/C++, Java, Groovy, e um pouco de Ruby e Python.
\end{itemize}

\item Rodrigo Gonçalves

\begin{itemize}
\item \textbf{Função:} Desenvolvedor
\item \textbf{Histórico:} 21 Anos, Cursando 8º Semestre de Engenharia de Software no campus Gama. Experiência com SDL e OpenGL na disciplina de Computação Gráfica. Conhecimento de linguagens C/C++, Java, PHP, Python e um pouco de Ruby on Rails.
\end{itemize}

\item Vinícius Corrêa de Almeida

\begin{itemize}
\item \textbf{Função:} Desenvolvedor
\item \textbf{Histórico:} 22 Anos, Cursando 9º Semestre de Ciência da Computação (Licenciatura) no campus Darcy Ribeiro. Ainda sem nenhuma experiência com o desenvolvimento de jogos. Conhecimento de linguagens C/C++, Prolog, Haskell, OpenGL, PHP, e um pouco de Ruby e Perl.
\end{itemize}

\item Heitor Campos

\begin{itemize}
\item \textbf{Função:} Artista
\item \textbf{Histórico:} 18 Anos, Cursando 3º Semestre Design no campus Darcy Ribeiro. Experiência com Photoshop e Ilustrator mas sem experiência no desenvolvimento de jogos. Bastante interessado na matéria de Design de Jogos buscando experiência com a área e trabalhos em equipe.
\end{itemize}

\item Max Von Behr

\begin{itemize}
\item \textbf{Função:} Artista
\item \textbf{Histórico:} 22 Anos, Cursando 8º Semestre Design no campus Darcy Ribeiro. Cursando a disciplina de Design de Jogos pela 3ª vez já adquiriu alguma experiência com esse desenvolvimento nos semestres passados e espera melhorar agora. Passou 2 anos na Lamparina Design (empresa júnior de Desenho Industrial) onde pode aprender bastante sobre gestão e coordenação de projetos. Artista 3D, adora ZBrush, Maya, textura e gráficos ultrarrealistas. Fã de RPGs e de boas narrativas. Participação no ultimo Game Jam. 
\end{itemize}

\item Aleph Telles de Andrade Casara

\begin{itemize}
\item \textbf{Função:} Recursos Humanos e Roteiro do Jogo
\item \textbf{Histórico:} 27 Anos, Cursando 7º Semestre Artes Cênicas no campus Darcy Ribeiro. Experiência em criação e direção de cena, construção de roteiro e personagem. Cursou um tempo de Jogos Digitais no Iesb.  
\end{itemize}

\item Washington Rayk

\begin{itemize}
\item \textbf{Função:} Músico
\item \textbf{Histórico:} 23 Anos, Cursando Comunicação Social - Audiovisuais no campus Darcy Ribeiro. Trabalha como ilustrador. Participou do Global Game Jam em 2012, 2013 e 2015 como artista e do Ludum Dare em 2014 como programador e artista de um jogo simples. Compositor da trilha de Knights of Pen and Paper  da Behold Studios a 2 anos atrás, e Co-Compositor da trilha do Chroma Squad (também da Behold), junto com o Raphael Müller.
\end{itemize}
\end{enumerate}

\end{document}
